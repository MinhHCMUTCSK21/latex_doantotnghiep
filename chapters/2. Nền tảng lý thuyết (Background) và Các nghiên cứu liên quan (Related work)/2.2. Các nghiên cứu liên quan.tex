\section{Các nghiên cứu liên quan (Related work)} 

\subsection{Tính cách và nghề nghiệp}
Hiện nay trên thế giới có rất nhiều công cụ khác nhau về lựa chọn nghề nghiệp dựa trên tính cách. MBTI, DISC và Holland là ba công cụ đánh giá tính cách phổ biến trên thế giới, được sử dụng để hiểu bản thân và những người khác. Mỗi công cụ có những điểm mạnh và điểm yếu riêng, và có thể hữu ích cho các mục đích khác nhau. Ngoài MBTI đã được giải thích ở phần trên
\textbf{DISC} là một công cụ đánh giá tính cách dựa trên bốn kiểu hành vi:
\begin{itemize}
    \item \textit{Dominance (D)}: Chi phối, quyết đoán
    \item \textit{Influence (I)}: Thuyết phục, hướng ngoại
    \item \textit{Steadiness (S)}: Ổn định, cẩn thận
    \item \textit{Conscientiousness (C):} Cẩn trọng, có trách nhiệm
\end{itemize}

DISC có thể được sử dụng để cải thiện giao tiếp, làm việc nhóm và lãnh đạo. Nó cũng có thể được sử dụng để xác định các điểm mạnh và điểm yếu của bản thân trong môi trường làm việc.

\textbf{Holland} là một công cụ đánh giá tính cách dựa trên lý thuyết của John Holland về sáu loại tính cách nghề nghiệp:
\begin{itemize}
    \item \textit{Thực tế (R)}: Thích làm việc với những thứ cụ thể và thực tế.
    \item \textit{Điều tra (I)}: Thích tìm kiếm thông tin và giải quyết vấn đề.
    \item \textit{Nghệ thuật (A)}: Thích làm việc với những thứ sáng tạo và thẩm mỹ.
    \item \textit{Xã hội (S):} Thích làm việc với mọi người và giúp đỡ người khác.
    \item \textit{Doanh nhân (E):} Thích lãnh đạo, thuyết phục và ảnh hưởng đến người khác.
    \item \textit{Kỹ thuật (T):} Thích làm việc với những thứ kỹ thuật và phức tạp.
\end{itemize}

Holland có thể được sử dụng để khám phá các lựa chọn nghề nghiệp phù hợp với tính cách và sở thích của bản thân. Nó cũng có thể được sử dụng để hiểu các điểm mạnh và điểm yếu của bản thân trong môi trường làm việc. 

A. Iwayemi. và những người đồng nghiên cứu đã dựa trên nội dung của MBTI về phân tích  tính cách để xây dựng một hệ thống hỗ trợ lựa chọn nghề nghiệp dựa trên 16 loại tính cách này được thực hiện bằng công cụ “\textbf{Prolog}” - một ngôn ngữ lập trình logic \cite{iwayemi}. Hệ thống trả lời câu hỏi và đưa ra danh sách các con đường nghề nghiệp có thể cho người dùng dựa trên tính cách của họ, ví dụ như Khoa học gia, Kỹ sư, Giáo sư, Bác sĩ, Nha sĩ, v.v. Hệ thống giúp giảm chi phí, giải quyết vấn đề nghề nghiệp và cung cấp lời khuyên nghề nghiệp đáng tin cậy không bị ảnh hưởng bởi lỗi con người. Đề xuất cho các nghiên cứu tương lai tập trung vào các mô hình khác có thể ảnh hưởng đến sự lựa chọn nghề nghiệp . 

Phát triển hơn nghiên cứu trên về lựa chọn nghề nghiệp dựa trên MBTI, Yizhou Zhou, Yong Zhang, Sijia Yu, Naijie Liu, đã phát triển \textbf{hệ thống gợi ý lộ trình nghề nghiệp} với sự hỗ trợ của trí tuệ nhân tạo AI dựa trên MBTI trong bối cảnh đa văn hóa, nhằm cân bằng giữa giá trị văn hóa đa dạng và hệ thống giáo dục \cite{yizhou}. Phương pháp tiếp cận của hệ thống là sự kết hợp công nghệ AI và phân tích dữ liệu lớn, sử dụng mô hình ngôn ngữ lớn của OpenAI Chat GPT để đề xuất đường lối nghề nghiệp phù hợp với loại MBTI, nền tảng học vấn, điểm số và sở thích của người dùng. Chức năng chính của hệ thống là thu thập thông tin cá nhân và kết quả MBTI của người dùng, tạo danh sách gợi ý nghề nghiệp dựa trên thuật toán AI, với sự chú trọng đặc biệt vào khả năng thích ứng đa văn hóa. Hệ thống cung cấp báo cáo phát triển nghề nghiệp chi tiết, giúp người dùng lập kế hoạch nghề nghiệp cá nhân hóa và hiểu sâu hơn về xu hướng phát triển nghề nghiệp toàn cầu. Đồng thời, hệ thống cũng hỗ trợ người dùng xác định kỹ năng và lĩnh vực kiến thức cần phát triển thêm. 

Một nghiên cứu khác được thực hiện bởi Dr. John Koti \textbf{phân tích ảnh hưởng của các đặc điểm tính cách MBTI đối với hành vi của nhân viên}, xem xét các biến số như đặc điểm dân số, kinh tế xã hội và sự hài lòng trong công việc \cite{drjohn}. Mục tiêu là hiểu mối quan hệ đáng kể giữa các biến số dân số xã hội, tình trạng việc làm và mức độ hài lòng trong công việc với các loại tính cách của nhân viên tại W.S Industries. Nghiên cứu đã tìm ra mối liên hệ đáng kể giữa các đặc điểm tính cách với giới tính, tôn giáo, tuổi tác, nơi cư trú và số năm làm việc, cũng như giữa chức danh và các đối lập tính cách như Trực giác - Cảm nhận và Cảm xúc - Tư duy. Điều này cho thấy vai trò quan trọng của các đặc điểm tính cách đối với hiệu suất làm việc của nhân viên.

Không chỉ ở trên thế giới, mà ở Việt Nam ta, ngay tại trường đại học Bách Khoa TP Hồ Chí MInh. Cũng đã có những nghiên cứu về MBTI và cách nó ảnh hưởng lên các ngành nghề như là nghiên cứu của Võ Đăng Khoa, Lê Hoài Long, Nguyễn Văn Châu, Đặng Ngọc Châu về \textbf{đặc điểm tính cách của kỹ sư Việt Nam}, nhằm hiểu rõ hơn về đặc điểm cá nhân và cải thiện việc tuyển dụng và sử dụng nguồn nhân lực hiệu quả trong ngành xây dựng \cite{khoa}.  Sử dụng công cụ KTS-II (Keirsey Temperament Sorter-II) để khảo sát và phân tích tính cách của kỹ sư xây dựng tại các vị trí khác nhau như tư vấn thiết kế, giám sát, thi công và quản lý dự án. Phân tích dựa trên 120 bảng khảo sát hợp lệ cho thấy đặc điểm tính cách nổi trội của nhóm kỹ sư thi công là ST (cảm giác - suy nghĩ). Kết quả nghiên cứu có thể giúp các doanh nghiệp xây dựng tối ưu hóa việc sử dụng nguồn nhân lực và giảm xung đột ngành nghề, qua đó nâng cao hiệu quả làm việc và cạnh tranh. Ngoài ra, đặc điểm tính cách có thể trở thành một cơ sở dự đoán hiệu quả lao động của kỹ sư, góp phần gia tăng hiệu  quả trong công việc thông qua việc phân công và tuyển dụng nhân sự phù hợp với vị trí việc làm.

Từ những nghiên cứu này, ta có thể thấy MBTI có sự ảnh hưởng đáng kể tới việc định hướng tương lai ngành nghề phù hợp của mỗi cá nhân, là một công cụ hữu ích giúp xác định dự đoán các biến số như sự sáng tạo và lựa chọn nghề nghiệp.

\subsection{Vikor và những ứng dụng}
Serafim Opricovic và Gwo-Hshiung Tzeng đã thực hiện một phân tích chi tiết và so sánh giữa hai phương pháp đưa ra quyết định đa tiêu chí là \textbf{VIKOR và TOPSIS}, thông qua các khía cạnh như hàm tổng hợp, kỹ thuật chuẩn hóa, và nguyên tắc cơ bản \cite{serafim}. Mặc dù cả hai phương pháp đều sử dụng khái niệm “gần gũi” với giải pháp lý tưởng, chúng lại có những cách tiếp cận khác nhau. Phương pháp VIKOR ưu tiên việc đáp ứng sở thích của đa số người dùng trong khi giảm thiểu sự hối tiếc cho người phản đối, còn phương pháp TOPSIS nhằm tìm ra giải pháp gần nhất với giải pháp tốt nhất và xa nhất với giải pháp tồi tệ nhất, mà không quan tâm rõ ràng đến mức độ quan trọng của sự gần gũi này.

Phương pháp VIKOR sử dụng chuẩn hóa tuyến tính để loại bỏ đơn vị của các hàm tiêu chí và xác định một giải pháp thỏa hiệp, cung cấp “lợi ích nhóm” tối đa cho “đa số” và tối thiểu hóa sự hối tiếc cá nhân cho “người phản đối”. Trong khi đó, phương pháp TOPSIS xác định một giải pháp có khoảng cách ngắn nhất đến giải pháp lý tưởng và khoảng cách xa nhất từ giải pháp tiêu cực, nhưng không xem xét đến tầm quan trọng tương đối của những khoảng cách này. Phân tích so sánh giữa hai phương pháp này được minh họa bằng một ví dụ số học, cho thấy sự tương đồng và một số khác biệt giữa chúng.

Trong quá trình đưa ra quyết định đa tiêu chí, việc lựa chọn giữa VIKOR và TOPSIS phụ thuộc vào bối cảnh cụ thể và yêu cầu của người ra quyết định. Cả hai phương pháp đều có những ưu điểm riêng và có thể được áp dụng tùy theo mục tiêu và tiêu chí cụ thể của dự án hoặc quyết định cần đánh giá. Điều quan trọng là phải hiểu rõ cách thức hoạt động và nguyên tắc của từng phương pháp để có thể sử dụng chúng một cách hiệu quả nhất.

Dựa vào nghiên cứu này, nghiên cứu của M. Aghajani Mir và cộng sự đã \textbf{áp dụng phương pháp VIKOR và TOPSIS để tối ưu hóa quản lý chất thải rắn đô thị}, cung cấp các giải pháp để lựa chọn hệ thống xử lý phù hợp nhất, xét đến các yếu tố môi trường và kinh tế. Kết quả nghiên cứu của họ chỉ ra rằng sự kết hợp của tái chế, tiêu hóa kỵ khí, bãi chôn lấp vệ sinh, chuyển đổi RDF và ủ phân là mô hình tối ưu cho quản lý chất thải tích hợp. Phương pháp TOPSIS được cải tiến để xếp hạng các phương pháp xử lý chất thải, trong khi phương pháp VIKOR được áp dụng để phân tích nhạy cảm. Kết quả cho thấy sự kết hợp của việc chôn lấp hợp vệ sinh (18.1\%), RDF (3.1\%), ủ phân (2\%), tiêu hóa kỵ khí (40.4\%), và tái chế (36.4\%) là mô hình tối ưu cho việc quản lý chất thải tích hợp. Nghiên cứu cung cấp khuyến nghị cho các nhà quản lý chất thải để cải thiện hệ thống quản lý chất thải thông qua cách tiếp cận bền vững hơn.

Một nghiên cứu khác của Ramkumar Yadav và những người khác áp dụng VIKOR trong vấn đè \textbf{xếp hạng vật liệu composite phục hồi răng (DRC)} \cite{sciencedirect1}. Phương pháp thực hiên bao gồm sử dụng phương pháp Entropy để tính trọng số cho từng tiêu chí ảnh hưởng đến quyết định lựa chọn vật liệu, sau đó tiếp tục sử dụng phương pháp VIKOR để xếp hạng các vật liệu DRC khác nhau dựa trên các tiêu chí được trọng số này. Lợi ích của nghiên cứu gồm Xem xét nhiều tiêu chí cùng một lúc, ngăn chặn bất kỳ yếu tố nào thống trị lựa chọn, cho phép so sánh vật liệu một cách hệ thống, phân tích điểm mạnh và điểm yếu của chúng trên nhiều thuộc tính khác nhau, dẫn đến việc ra quyết định sáng suốt bằng cách đánh giá nhiều lựa chọn vật liệu tổng hợp, cuối cùng dẫn đến việc phục hồi răng tốt hơn và sự hài lòng cao hơn của bệnh nhân. Nghiên cứu chứng minh thành công ứng dụng Entropy-VIKOR kết hợp để xếp hạng vật liệu DRC. Trọng số của từng tiêu chí (ví dụ: độ bền, khả năng tương thích sinh học) được tính toán bằng phương pháp Entropy. VIKOR sau đó được sử dụng để xếp hạng các composite răng khác nhau dựa trên các tiêu chí được trọng số này. Nghiên cứu kết luận rằng DHZ6 là vật liệu composite răng được xếp hạng cao nhất trong số các lựa chọn được điều tra. Nghiên cứu này giới thiệu một phương pháp mới để xếp hạng vật liệu phục hồi răng bằng cách sử dụng kết hợp các kỹ thuật MCDM. Các phát hiện có tiềm năng hướng dẫn nha sĩ lựa chọn vật liệu phù hợp nhất cho nhu cầu cụ thể của từng bệnh nhân .

Một nghiên cứu khác về \textbf{nghiên cứu rủi ro của xe tự hành} do Bhosale Akshay Tanaji, Sayak Roychowdhury sử dụng Phương pháp VIKOR tích hợp của BWM sử dụng bộ mờ trung tính để đánh giá rủi ro an ninh mạng của các phương tiện tự động và được kết nối \cite{sciencedirect2}. Về các vấn đề Xe tự hành (CAV) dễ bị tấn công mạng từ nhiều loại kẻ thù, chưa có phương pháp đánh giá rủi ro của từng viruss. Quyết định phòng thủ an ninh mạng phụ thuộc vào đánh giá của chuyên gia, thiếu tính khách quan, thiếu sự cân nhắc giữa các tiêu chí. Với phương pháp nghiên cứu bao gồm xác định các loại kẻ thù tấn công an ninh mạng của CAV từ đó thu thập ý kiến chuyên gia về mức độ nghiêm trọng và tần suất tấn công của các loại kẻ thù sau đó sử dụng lý thuyết tập hợp mờ số trung tính một giá trị (SVNFS) để xử lý tính chủ quan trong đánh giá của chuyên gia và dùng phương pháp BWM để xác định trọng số cho các tiêu chí đánh giá cuối cùng dùng phương pháp VIKOR để xếp hạng mức độ rủi ro của các loại kẻ thù. Nghiên cứu này đã xác định các loại kẻ thù tấn công an ninh mạng của CAV, xây dựng mô hình đánh giá rủi ro dựa trên lý thuyết tập hợp mờ. Đề xuất phương pháp kết hợp BWM, VIKOR và SVNFS để đánh giá rủi ro của các loại kẻ thù, phân tích nhạy cảm để kiểm tra tính tin cậy của phương pháp cuối cùng so sánh phương pháp đề xuất với các phương pháp khác. Đem đến kết quả khả quan như là nghiên cứu xếp hạng mức độ rủi ro của các loại kẻ thù tấn công CAV, cung cấp thông tin cho nhà sản xuất CAV và nhà quản lý giao thông để xây dựng chiến lược phòng thủ an ninh mạng. Nghiên cứu này đã đóng góp vào việc phát triển các phương pháp đánh giá rủi ro an ninh mạng cho xe tự hành. Phương pháp đề xuất có tiềm năng được áp dụng rộng rãi trong thực tế để đảm bảo an ninh mạng cho CAV .

Nghiên cứu của C.M. La Fata, A. Giallanza, R. Micale, G. La Scalia về \textbf{xếp hạng rủi ro an toàn, sức khỏe nghề nghiệp theo góc độ đa tiêu chí}: Bao gồm yếu tố con người và áp dụng VIKOR \cite{sciencedirect3}. Ngày nay, an toàn sức khỏe nghề nghiệp (OHS) được công nhận ngày càng quan trọng trong việc quản lý và cải thiện liên tục bởi mọi tổ chức. OHS nhằm ngăn ngừa thương tích và bệnh tật cho người lao động, từ đó tác động tích cực đến năng suất, khả năng cạnh tranh, uy tín và tiết kiệm chi phí.  Quản lý OHS dựa trên kết quả đánh giá rủi ro để xác định các biện pháp khắc phục nhằm giảm thiểu rủi ro xuống mức chấp nhận được. Ma trận rủi ro (dựa trên Xác suất xảy ra và Mức độ nghiêm trọng) là phương pháp đánh giá bán định lượng, dễ dàng thực hiện nhưng có nhiều hạn chế. Xác suất và Mức độ nghiêm trọng được coi trọng ngang nhau, sử dụng phép nhân để tính toán mức độ rủi ro không phân biệt được các sự kiện khác nhau. Sai sót của con người là nguyên nhân gây ra 60-80\% tai nạn nơi làm việc nhưng không được tính toán đầy đủ trong Ma trận Rủi ro. Nghiên cứu đề xuất phương pháp MCDM để khắc phục hạn chế của Ma trận Rủi ro. Ba tiêu chí đánh giá được sử dụng bao gồm : Xác suất xảy ra, Mức độ nghiêm trọng và quan trọng nhất Đóng góp của yếu tố con người (sử dụng kết hợp kỹ thuật HEART và SPAR-H). Trọng số tương đối của các tiêu chí được đánh giá bằng Analytic Hierarchy Process (AHP). Phương pháp Vlse Kriterijumska Optimizacija Kompromisno Resenje (VIKOR) được sử dụng để xếp hạng rủi ro. Phương pháp MCDM được áp dụng cho một doanh nghiệp sản xuất đồ gỗ ở Sicily. Kết quả so sánh với Ma trận Rủi ro truyền thống cho thấy phương pháp MCDM mới có khả năng phân biệt rủi ro tốt hơn. Nghiên cứu đề xuất phương pháp MCDM mới để đánh giá rủi ro OHS, tính đến cả yếu tố con người. Phương pháp này linh hoạt, dễ dàng áp dụng cho các ngành công nghiệp khác nhau. Nghiên cứu cung cấp cho doanh nghiệp công cụ để ưu tiên các rủi ro và lựa chọn các biện pháp khắc phục phù hợp.

Quản lý chuỗi cung ứng đóng vai trò quan trọng trong việc giảm thiểu rủi ro, giá thành, đồng thời gia tăng lợi nhuận cho doanh nghiệp. Lựa chọn nhà cung cấp phù hợp là một trong những hoạt động then chốt của quản lý chuỗi cung ứng, giúp giảm chi phí vận hành và cải thiện chất lượng sản phẩm. Bài báo Xiao-Yue You, Jian-Xin You, Hu-Chen Liu, Lu Zhen \textbf{nghiên cứu phương pháp VIKOR mở rộng để lựa chọn nhà cung cấp theo nhóm}, với thông tin ngôn ngữ dạng cặp nhị nguyên khoảng. Toàn cầu hóa và thay đổi công nghệ khiến việc lựa chọn nhà cung cấp phù hợp trở nên cần thiết do đó  việc quyết định lựa chọn nhà cung cấp ảnh hưởng đến chi phí, chất lượng sản phẩm, khả năng cạnh tranh và lợi nhuận của doanh nghiệp. Bài báo điểm qua một số phương pháp thường dùng như AHP, TOPSIS, DEA, DEMATEL, lập trình tuyến tính (LP). Ngoài ra, lý thuyết tập fuzzy được sử dụng để xử lý tính mơ hồ và không rõ ràng trong quá trình ra quyết định. Bài báo đề xuất phương pháp VIKOR mở rộng sử dụng biến ngôn ngữ dạng cặp nhị nguyên khoảng để xử lý tính không chắc chắn trong đánh giá nhà cung cấp. Ưu điểm của phương pháp này là cho phép người ra quyết định linh hoạt thể hiện đánh giá bằng các thuật ngữ ngôn ngữ khác nhau, đồng thời tính đến khoảng tin cậy. Phương pháp VIKOR mở rộng với biến ngôn ngữ dạng cặp nhị nguyên khoảng là phương pháp linh hoạt và chính xác hơn để giải quyết bài toán lựa chọn nhà cung cấp trong môi trường thông tin không đầy đủ và không chắc chắn.

Một nghiên cứu khác của Muhammad Saqlain về sản xuất hydro bền vững: \textbf{Phương pháp ra quyết định sử dụng VIKOR và Bộ Hypersoft trực quan} \cite{muhammad}. Nghiên cứu này tập trung vào việc lựa chọn phương pháp sản xuất hydro đúng đắn trong bối cảnh năng lượng bền vững, sử dụng phương pháp quyết định đa tiêu chí VIKOR và bộ Intuitionistic Hypersoft Sets (IHSSs). Bằng cách áp dụng IHSSs, nghiên cứu đã giải quyết được sự không chắc chắn, tối ưu hóa việc lựa chọn công nghệ, phân bổ nguồn lực và đánh giá hậu quả môi trường. Phương pháp VIKOR được sử dụng để đánh giá hệ thống sản xuất hydro dựa trên các tiêu chí khác nhau và xếp hạng các phương án thay thế để tìm ra giải pháp tối ưu. Nghiên cứu cung cấp cái nhìn sâu sắc về sức mạnh và điểm yếu của từng phương pháp sản xuất hydro từ góc độ kỹ thuật và bền vững, hỗ trợ việc ra quyết định thông tin. Nó cũng mở ra hướng nghiên cứu mới trong việc áp dụng các phương pháp như AHP và TOPSIS trong bối cảnh neutrosophic.
Trong lĩnh vực lựa chọn ngành nghề học của đồ án này, nghiên cứu của Rama MALLICK về MAGDM trung tính dựa trên sử dụng chiến lược CRITIC-EDAS kết hợp với phương pháp toán tử tổng hợp hình học \cite{rama}. Mục tiêu Nghiên Cứu là \textbf{phát triển phương pháp MAGDM dựa trên chiến lược CRITIC-EDAS sử dụng toán tử tổng hợp hình học trong môi trường SVNS}. Trong ứng dụng thực tế bài báo mô tả việc áp dụng phương pháp cho vấn đề lựa chọn nghề nghiệp của sinh viên ngành thương mại. Bài báo cũng sử dụng một phương pháp độc đáo đó là sử dụng toán tử tổng hợp hình học CRITIC-EDAS, một phương pháp chưa từng được ghi nhận trong nghiên cứu khoa học trước đây. Đem đến 1 kết quả đáng kinh ngạc khi phương pháp giúp xử lý thông tin không chắc chắn và mơ hồ, cung cấp một công cụ hữu ích cho việc đưa ra quyết định trong các tình huống có nhiều thuộc tính xung đột.

Hoặc nghiên cứu của Rekha Sahu, Satya R. Dash and Sujit Das \cite{rekha}. Bài báo này tập trung vào việc \textbf{lựa chọn nghề nghiệp cho sinh viên sử dụng phương pháp kết hợp giữa tập hợp mờ ảnh (PFS) và lý thuyết tập hợp thô (RS)}, để xử lý thông tin không chắc chắn liên quan đến việc lựa chọn nghề nghiệp của sinh viên. Đề xuất sử dụng hai phép đo khoảng cách kết hợp dựa trên khoảng cách Hausdorff, Hamming và Euclidean trong môi trường mờ ảnh “fuzzy”, cùng với một cách tiếp cận thuật toán sử dụng các phép đo này và lý thuyết RS. Thực hiện hai nghiên cứu điển hình để kiểm chứng tính khả dụng của ý tưởng đề xuất, giúp quản lý các tình huống không nhất quán khi lựa chọn ngành học cho sinh viên. Bài báo kết luận rằng việc kết hợp PFS và RS có thể giúp giải quyết hiệu quả các vấn đề liên quan đến việc lựa chọn nghề nghiệp cho sinh viên, đặc biệt trong các tình huống có sự không chắc chắn hoặc mâu thuẫn.

Trực tiếp hơn trong việc sử dụng trực tiếp VIKOR, bài nghiên cứu của Kelvin Ade Wizura, Jusuf Wahyudi , Juju Jumadi về việc áp dụng phương pháp VIKOR (VIšekriterijumsko KOmpromisno Rangiranje) để \textbf{đề xuất lựa chọn chuyên ngành cho học sinh tại SMA Negeri 6 Bengkulu Tengah} \cite{kelvin}. Phương pháp này giúp xếp hạng các lựa chọn dựa trên một số tiêu chí nhất định và tìm ra giải pháp gần với lý tưởng nhất. Học sinh thường gặp khó khăn trong việc lựa chọn ngành học do ảnh hưởng từ ba yếu tố: ý kiến của phụ huynh, xu hướng ngành học hiện tại và thành tích học tập. Bài báo đã kết hợp phương pháp VIKOR giúp xếp hạng các lựa chọn dựa trên tiêu chí nhất định, giúp học sinh có được quyết định chính xác hơn. Kết quả thử nghiệm hệ thống cho thấy phương pháp VIKOR có thể xếp hạng học sinh một cách hiệu quả, với học sinh có điểm số thấp nhất nhận được xếp hạng cao nhất. Hệ thống cũng cho phép tính toán và xếp hạng dựa trên trọng số tiêu chí, đảm bảo kết quả phù hợp với đánh giá thủ công.

Từ các nghiên cứu trên chúng ta có thể thấy rõ sự mạnh mẽ của VIKOR cũng như tính linh hoạt có thể áp dụng cho nhiều, loại hình bài toán khác nhau, trong bài đồ án này nhóm tác giả sử dụng vikor như là một công cụ chính giúp cho người dùng hướng tới lựa chọn giải pháp cân bằng giữa các yếu tố và lựa cũng như tự khai phá, tìm hiểu ra ikigai của riêng mình.

\subsection{Một số hệ gợi ý, hỗ trợ ra quyết định ngành học khác}
Hệ thống khuyến nghị để \textbf{lựa chọn chương trình đại học phù hợp} tại các cơ sở giáo dục đại học \textbf{sử dụng dữ liệu sinh viên sau đại học} của nhóm tác giả Yara Zayed, Yasmeen Salman, Ahmad Hasasneh \cite{yara}. Với mục tiêu nghiên cứu là  xây dựng một hệ thống đề xuất thông minh giúp sinh viên chọn chương trình đại học phù hợp dựa trên hiệu suất học tập trước đó và dữ liệu thị trường lao động. Sử dụng các kỹ thuật học máy giám sát như Cây Quyết định(Decision Tree), Rừng Ngẫu nhiên(Random Forest) và Máy Vector Hỗ trợ (Support Vector Machine) để dự đoán chuyên ngành đại học. Các tiêu chí(Label) liên quan đến lịch sử học thuật của sinh viên và thị trường việc làm được sử dụng làm đầu vào cho mô hình. Ngoài ra việc phân tích tầm quan trọng của các tiêu chí cho thấy rằng tỷ lệ học vị, tỷ lệ MBA, và kết quả bài kiểm tra đầu vào là những đặc trưng đóng góp chính cho mô hình. Kết quả cho thấy phương pháp của họ vượt trội so với nghiên cứu đã công bố trước đó, với Rừng Ngẫu Nhiên (Random Forest) đạt độ chính xác 97.70\% so với 75.00\%. Tuy 97,07\% là một con số rất lớn nhưng việc điều chỉnh các biến ngẫu nhiên quá phù hợp với mô hình có thể dẫn đến tình trạng "overfitting", khiến mô hình hoạt động hiệu quả trên tập dữ liệu huấn luyện nhưng không chính xác trên tập dữ liệu mới hoặc hệ thống có thể thiếu dữ liệu về một số chương trình đại học hoặc thị trường lao động ở một số khu vực nhất định. Tuy vậy hệ thống cũng có những ứng dụng to lớn của riêng mình. Hệ thống đề xuất này có thể hỗ trợ sinh viên chọn chuyên ngành phù hợp dựa trên hiệu suất học tập trước đó và dữ liệu thị trường việc làm. Cải thiện kết quả từ nghiên cứu đã công bố trước đó bằng cách áp dụng kỹ thuật điều chỉnh siêu tham số, mà nghiên cứu trước không có. Hệ thống đề xuất này không chỉ giúp sinh viên đưa ra quyết định chính xác hơn khi chọn chuyên ngành mà còn góp phần vào việc nâng cao chất lượng giáo dục đại học.

Hệ thống đề xuất thích ứng sử dụng thuật toán học máy để \textbf{dự đoán chương trình học tập tốt nhất của học sinh} do nhóm tác giả Ayman Elshenawy Elsefy, Mohamed Ezz \cite{mohamed} với mục tiêu nghiên cứu là Xây dựng hệ thống đề xuất thích ứng dựa trên dữ liệu khai thác từ hành vi học tập của sinh viên trong năm chuẩn bị để dự đoán con đường giáo dục phù hợp. Mô hình đề xuất có khả năng chọn lựa thuật toán học máy tốt nhất cho mỗi bộ phận của trường đại học, tìm ra dữ liệu quan trọng trong quá trình đề xuất và đề xuất chính xác chương trình kỹ thuật phù hợp cho sinh viên. Hệ thống sử dụng các kỹ thuật khai thác dữ liệu để tự động áp dụng các phương pháp khác nhau cho việc trích xuất đặc trưng và xây dựng mô hình phù hợp cho từng lĩnh vực giáo dục. Vấn đề được định hình như một bài toán phân loại đa nhãn đa lớp và dữ liệu được chuyển đổi tự động thành phân loại nhị phân một-chọn-tất cả. Kết quả thu được cho thấy mô hình đề xuất có thể đề xuất thuật toán học máy (mô hình) tốt nhất cho từng bộ phận của khoa, tìm ra dữ liệu liên quan quan trọng trong quá trình đề xuất và đề xuất sinh viên với bộ phận kỹ thuật phù hợp với độ chính xác cao. Phương pháp và kết quả được đánh giá dựa trên hiệu suất của các mô hình học máy khác nhau và việc lựa chọn đặc trưng liên quan cho mỗi bộ phận giáo dục cụ thể. Điều này giúp tối ưu hóa quá trình đề xuất và cải thiện khả năng dự đoán thành công của sinh viên trong các ngành kỹ thuật khác nhau. Kết lại, Mô hình đề xuất có khả năng chọn lựa thuật toán học máy tốt nhất cho mỗi bộ phận của trường đại học, tìm ra dữ liệu quan trọng trong quá trình đề xuất và đề xuất chính xác chương trình kỹ thuật phù hợp cho sinh viên. 

Tương tự vậy các tác giả trong nghiên cứu “Using Recommender Systems for Matching Students with Suitable Specialization: An Exploratory Study at King Abdulaziz University” đã phát triển \textbf{Hệ thống Đề xuất Đại học King Abdelaziz} (KAURS) \cite{khloud}, một hệ thống đề xuất để dự đoán và gợi ý chuyên ngành phù hợp cho sinh viên dựa trên khả năng và điểm số của họ trong năm chuẩn bị. Trong nghiên cứu này, thuật toán KNN đã được sử dụng để dự đoán chuyên ngành thích hợp. Quá trình xác thực cho hệ thống được thực hiện bằng phương pháp gấp k-fold, với độ chính xác đạt 74.79\%.

Ngoài ra, các nhà nghiên cứu bao gồm Stein, S.A.; Weiss, G.M.; Chen, Y.; Leeds, D.D trong nghiên cứu A College Major Recommendation System \cite{researchgate} đã đề xuất một hệ thống đề xuất nhằm cải thiện kết quả học tập của sinh viên bằng cách \textbf{gợi ý một số chuyên ngành phù hợp (n) sử dụng phương pháp KNN}; các nhà nghiên cứu đã đo lường tỷ lệ sinh viên có chuyên ngành của họ là chuyên ngành được đề xuất (dựa trên sinh viên có các khóa học và hiệu suất tương tự) bằng cách sử dụng khoảng cách cosine điều chỉnh. Tuy nhiên, điều này không thể xác định liệu chuyên ngành có phù hợp với sinh viên hay không; để xác nhận điều này, một phép đo khác được sử dụng để kiểm tra xem hiệu suất của sinh viên có ở mức trên hoặc bằng hiệu suất trung bình trong chuyên ngành này. Hệ thống đạt được độ chính xác là 67\%.

Một nghiên cứu khác về gợi ý ngành học của nhóm tác giả Tajul Rosli Razak1, Muhamad Arif Hashim, Noorfaizalfarid Mohd Noor, Iman Hazwam Abd Halim, Nur Fatin Farihin Shamsul về \textbf{hệ thống khuyến nghị lựa chọn nghề nghiệp sử dụng lập luận mờ (fuzzy logic)} để hỗ trợ sinh viên Đại học Teknologi MARA (UiTM) Perlis, Malaysia, trong việc chọn lựa nghề nghiệp phù hợp với kỹ năng và khả năng của họ \cite{tajul}. Hệ thống nhằm cung cấp hướng dẫn và khuyến nghị nghề nghiệp dựa trên kết quả bài kiểm tra nghề nghiệp của sinh viên, bởi vì sinh viên thường gặp khó khăn trong việc lựa chọn nghề nghiệp do thiếu kinh nghiệm và sự hỗ trợ từ người thân, giáo viên hoặc tư vấn nghề nghiệp. Do đó hệ thống sử dụng logic mờ “fuzzy logic” để hỗ trợ sinh viên UiTM trong việc chọn lựa con đường nghề nghiệp. Logic mờ là một hình thức logic đa giá trị hoặc logic xác suất, phù hợp hơn so với các quy tắc cố định và chính xác. Quy trình ánh xạ bao gồm các hàm thành viên đầu vào và đầu ra, các toán tử logic mờ, các quy tắc nếu-thì mờ, tổng hợp các tập hợp đầu ra, và quá trình làm rõ. Sau đó, Fuzzification chuyển đổi các biến đầu vào thành các giá trị phù hợp trong miền của cuộc thảo luận, trong khi defuzzification chuyển đổi các biến đầu ra mờ thành một giá trị rõ ràng duy nhất. Hệ thống này bao gồm sự tham gia của cố vấn học đường như một chuyên gia trong lĩnh vực để giúp thu thập dữ liệu về kỹ năng và tính cách cho mỗi lựa chọn nghề nghiệp. Kết quả đạt được khi hệ thống giúp sinh viên tự kiểm tra mà không cần sự hỗ trợ toàn diện từ cố vấn, đồng thời đánh giá sức mạnh kỹ năng, khả năng và đặc điểm tính cách của họ để đề xuất các lựa chọn nghề nghiệp có thể.

Một nghiên cứu khác về tìm kiếm việc làm của nhóm tác giả Suleiman Ali Alsaif , Minyar Sassi Hidri, Imen Ferjani, Hassan Ahmed Eleraky and Adel Hidri về \textbf{hệ thống đề xuất song phương dựa trên NLP} để hỗ trợ cả người tìm việc và nhà tuyển dụng \cite{suleiman}. Hệ thống này giúp kết nối ứng viên với các vị trí công việc phù hợp và ngược lại.  Sử dụng NLP để phân tích và xử lý thông tin từ cả hồ sơ xin việc và mô tả công việc, nhằm tìm ra sự tương đồng và khớp nối giữa chúng. Hệ thống được đánh giá qua một bộ dữ liệu hồ sơ xin việc/mô tả công việc, với mục tiêu cải thiện quá trình tuyển dụng máy tính hóa và giảm thiểu thất nghiệp. Hệ thống bao gồm việc thu thập dữ liệu từ trang web sa.indeed.com.Tiền xử lý dữ liệu bằng cách xử lý dữ liệu bằng cách chuyển đổi dữ liệu đã gắn thẻ từ Dataturks sang định dạng spaCy, loại bỏ khoảng trắng đầu và cuối từ các phạm vi thực thể. Huấn luyện mô hình nhận dạng thực thể có tên (NER) sử dụng spaCy. Và tạo danh sách công việc và hồ sơ, sau đó tiến hành so khớp để tìm công việc phù hợp nhất cho người tìm việc và hồ sơ phù hợp nhất cho nhà tuyển dụng. Mô hình được đánh giá dựa trên các chỉ số hỗ trợ quyết định như độ chính xác, độ chính xác dự đoán và F1-score. Kết quả thu được xác nhận rằng hệ thống đề xuất có thể giải quyết vấn đề đề xuất hai chiều và cải thiện độ chính xác dự đoán, tuy vậy độ chính xác gần như xấp xỉ 1 lại đôi khi nói lên vấn đề về khả năng overfitting của hệ thống này, tuy vậy chúng tôi chỉ nghiên cứu hướng thực hiện và bỏ qua những thiếu sót của nghiên cứu này. Nhưng tổng kết lại kết quả đạt được của nghiên cứu vẫn rất khả quan khi  cho thấy hệ thống đề xuất có khả năng cải thiện độ chính xác trong việc dự đoán và đề xuất.

Trong các phần trước chúng tôi đã đề cập tới mô hình Ikigai về hỗ trợ sinh viên lựa chọn nghề nghiệp. Xin được nhắc lại một lần nữa, IKigai là một khái niệm Nhật Bản giúp tìm kiếm mục đích sống của cá nhân, hay lí do để tồn tại. Xuất phát từ điều này, nhóm tác giả Ángel Millán, Jorge García-Unanue, Marta Retamosa đã phát triển một\textbf{ nền tảng trực tuyến dựa trên phương pháp Ikigai để hỗ trợ học sinh chọn nghề nghiệp phù hợp} \cite{angel}. Phương pháp Ikigai được thực hiện thông qua một quá trình tuần tự, hợp tác với các cố vấn trường trung học, Học sinh hoàn thành 4 bảng câu hỏi trong giai đoạn phân tích dự đoán, liên kết với 17 lĩnh vực kiến thức đã được thiết lập, từ đó nền tảng trực tuyến tạo ra báo cáo cá nhân cho mỗi học sinh, biểu diễn điểm số liên quan đến 15 lĩnh vực kiến thức. Cuối cùng, dựa trên báo cáo, trường đại học gửi đề xuất các khóa học đại học có thể phù hợp với hồ sơ nghề nghiệp của học sinh qua email/ Phương pháp này giúp học sinh cải thiện nhận thức bản thân và hiệu quả trong quá trình đưa ra quyết định, đồng thời hỗ trợ cố vấn trường học trong công việc của họ. Cũng như đem đến lợi ích to lớn khi cải thiện nhận thức bản thân người dùng và hiệu quả quyết định nghề nghiệp cho học sinh, đồng thời hỗ trợ công tác hướng nghiệp của trường học.

Hệ thống của Ángel Millán, Jorge García-Unanue, Marta Retamosa chỉ trả lời dựa trên các lựa chọn có sẵn của người dùng, điều này có thể dẫn đến thiên kiến hoặc tình trạng "\textit{overfitting}" (học máy quá khớp với dữ liệu). Chúng tôi nhận thấy đây không phải là cách tiếp cận tốt, do đó chúng tôi áp dụng các phương pháp Quyết định đa tiêu chuẩn (Multi-Criteria Decision Making) để đưa ra một cách tiếp cận khách quan hơn. Trong nghiên cứu này, chúng tôi sử dụng phương pháp Tổng hợp Theo Hệ Số Trọng Điểm (Weight-Sum) và phương pháp VIKOR trong một hệ thống hỗ trợ quyết định. Cách tiếp cận này cho phép sinh viên xem xét tầm quan trọng tương đối của các yếu tố khác nhau và tìm ra giải pháp thỏa hiệp phù hợp với Ikigai của họ, đây là trọng tâm chính của nghiên cứu. Chúng tôi cũng xin cảm ơn các nghiên cứu khác đã giúp chúng tôi có cái nhìn tổng quan, toàn cảnh hơn cách áp dụng các phương pháp vào bài toán gợi ý lựa chọn ngành học giành cho sinh viên từ đó có thể lựa chọn được hướng đi của riêng cá nhân nhóm trong đồ án này.