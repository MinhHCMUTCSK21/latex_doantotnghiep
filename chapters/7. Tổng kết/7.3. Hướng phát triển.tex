\section{Định hướng phát triển}
   Với sứ mệnh mang lại giải pháp tối ưu cho việc lựa chọn ngành nghề và sự nghiệp cho người dùng, hệ thống Bkareer không ngừng phát triển và hoàn thiện để đáp ứng được nhu cầu ngày càng đa dạng và phức tạp của người dùng. Tiếp sau đây, nhóm sẽ đề cập đến những định hướng phát triển trong tương lai của Bkareer, bao gồm những nỗ lực để cải thiện giao diện, mở rộng các tính năng, và tối ưu hóa hiệu suất hệ thống. 
   \begin{itemize}
       \item \textit{Hoàn thiện giao diện:} Tiếp tục tối ưu và cải thiện giao diện người dùng để đảm bảo trải nghiệm người dùng tốt nhất. Tích hợp các phản hồi từ người dùng để điều chỉnh và cải thiện giao diện, đồng thời tuân thủ các chuẩn đánh giá giao diện như Google Lighthouse để đạt được hiệu suất tối ưu.
       \item \textit{Tối ưu theo chuẩn SEO và chuẩn đánh giá giao diện (Google Lighthouse)}: Tối ưu hóa hệ thống để tăng cường khả năng tìm thấy trên các công cụ tìm kiếm và cải thiện trải nghiệm người dùng. Điều này có thể bao gồm việc tối ưu hóa các từ khóa, cải thiện tốc độ tải trang và tuân thủ các nguyên tắc thiết kế giao diện hiện đại.
       \item \textit{Thử nghiệm triển khai thêm một số giải pháp khác về MCDM} để so sánh kết quả: Nghiên cứu và thử nghiệm các phương pháp và giải pháp khác về Multi-Criteria Decision Making (MCDM) để so sánh và đánh giá hiệu quả của chúng. Điều này có thể giúp tìm ra giải pháp tối ưu nhất cho hệ thống Bkareer.
       \item \textit{Tiến hành khảo sát toàn diện} hơn để có những bước đánh giá, phát triển phù hợp hơn: Tăng cường quá trình khảo sát và thu thập phản hồi từ người dùng để hiểu rõ hơn về nhu cầu và mong muốn của họ. Dựa vào thông tin này để phát triển và điều chỉnh hệ thống một cách phù hợp và hiệu quả hơn.
   \end{itemize}

Nghiên cứu này giới thiệu một phương pháp mới để phát triển hệ thống hỗ trợ giúp sinh viên lựa chọn ngành học. Khác biệt so với các phương pháp hiện có, khung kiến trúc đề xuất kết hợp triết lý cân bằng cuộc sống dựa trên khái niệm Ikigai ngàn năm tuổi của Nhật Bản. Bằng cách tích hợp các kỹ thuật Phân tích Quyết định Đa tiêu chí MCDA hoặc MCDM với VIKOR và trọng số tổng hợp, nghiên cứu này nhằm mục đích đóng góp một góc nhìn mới cho các lĩnh vực tư vấn học thuật, đề xuất ngành học và hệ thống hỗ trợ quyết định.

Đối với các nghiên cứu trong tương lai, chúng tôi tin rằng mặc dù mục đích của mô hình Ikigai là hướng tới một cuộc sống cân bằng giữa đam mê, sở trường, nhu cầu xã hội và thu nhập, nhưng mỗi quốc gia, dân tộc và nền văn hóa đều có đặc điểm riêng. Do đó, nghiên cứu trong tương lai nên tập trung vào việc xác định trọng số tối ưu cho từng khu vực. Hơn nữa, việc triển khai hệ thống trên quy mô lớn hơn là cần thiết để tiến hành đánh giá toàn diện hơn về hiệu quả của nó.